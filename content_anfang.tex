\section{Einleitung}

Fahrzeug-assiertes Fahren mit zusätzlichen AR-Features für eine virtuelle Test- und Simualtionsumgebung für die Darstellung von 3D-Inhalten 

\ldots 

\subsection{Aufgabenbeschreibung}
Brille Kalibrieren, damit bei der durchsicht durch eine Brille, das wahrgenommene 3D der simulierten dargestellten Realität

\subsection{Motivation}
Im Kontext Kognitiver Automibile hat eine Kalibrierte 3D Sicht ebenfalls eine große Bedeutung: Hierbei gibt es zwei große Anwendungsgebiete, einerseits können hiermit neue Geräte oder Anzeigen Simuliert werden, andererseits kann hiermit aber auch eine Simulierte Realität dem Sicherheitsfahrer angezeigt werden. So wird z.B. ein Auffahrbremsen simuliert, so kann dem Fahrer das Simulierte Auto angeziegt werden. Oder wird in eine simulierte Parkbuch eingeparkt, so kann diese vom Sicherheitsfahrer durhc die Brille beurteilt werden.

\subsection{Aufteilung}
Die Projektgruppe hat sich darauf geeinigt, dass jeder den Teil dokumentiert, den er in der Präsentation wiedergegeben hat.