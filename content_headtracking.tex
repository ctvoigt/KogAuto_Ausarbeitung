\section{Headtracking}

\subsection{Senorik}
\subsubsection{Gyroskop}
Funktionsweise
Tiefpassfilter
Bias-Korrektur
HBW \& LBW Fusion
Wertebereiche, wie alte Präsentation
Integration
Eigenschaften (schnell, aber Drift durch Integration)
\subsubsection{Beschleunigungssensor}
Funktionsweise
für Roll und Pitch
Tiefpassfilter
Umrechnung auf m/s$^2$
Berechnung von Roll und Pitch direkt aus Werten
Problem: ungenügendes Yaw
\subsubsection{Magnetometer}
Funktionsweise
Tiefpassfilter
Magnetometer-Bias-Korrektur (durch Min/Max-Werte), Normierung auf [-1,1]
Magnetometer-Kreisplots vor/nach Kalibrierung, Magnetometer-Fehler in Kreisplots
Transformation auf XY Ebene (Algorithmus Ralf)
\subsubsection{Kamera}

\subsection{Sensordaten-Fusion}
Ziel: Orientierung
\subsubsection{Betrachtung verschiedener Algorithmen}
Madgwick (Grund dagegen: Optimierungsverfahren basierend auf Quaternionendarstellung dadurch schwer nachvollziehbar/debuggbar), Kalman, Komplementärfilter Entscheidung für Komplementärfilter: gut genug für unsere Zwecke, leicht anpassbar, Fehlerverteilungen für Sensoren unbekannt (für Kalman nötig)
\subsubsection{Komplementär-Filter}
Blockschaltbild der Fusion Gewichtung: Funktionsweise formal: “gyro.slerp(0.05, acc).slerp(0.01, mag)” Zwischenfazit Insgesamt eingeständiger Ansatz (offline)
\subsubsection{Verarbeitungsabfolge}
(Blockdiagramm und Erklärung)

\subsection{Marker-Tracking}
\subsubsection{Motivation}
Stützung relativ zum Fahrzeug (Korrektur) Verwendung von ALVAR zur Marker-Erkennung Berechnung der Kopf-Pose aus Maker-Orientierung Verrechnung mit Komplementärfilter
\subsubsection{Kinect}
Probleme mit Erkennung des Kopfes (Brille passt nicht zur Kopfmaske) Mit der Bille geht es insgesamt nicht so gut Verarbeitungspipeline ist langsam
\subsubsection{Umsetzung}

\subsection{Zusammenspiel der Ansätze}
\subsubsection{Koordinatensysteme}
(Bild mit allen TFs und deren Abh.) Erklärung der verschiedenen Pfade (Mit/Ohne Marker..)
\subsubsection{Loosely-Coupled-System}
Verrechnung von Auto-IMU-Daten und Brillen-IMU-Daten Zur Bestimmung der Sensordaten in Autokoordinaten Es fehlt nur noch die Integration

