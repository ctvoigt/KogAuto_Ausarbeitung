
\subsection{Sensordaten-Fusion}
\label{headtracking_fusion_subsec}
Die Fusionierung beschreibt die Verknüpfung von mehreren Sensoren um Informationen besserer Qualität zu gewinnen. In diesem Projekt gilt des die aufbereiteten Daten aus den beiden Gyroskopen, Accelerometer und Magenetometer derart zu fusioneieren, dass letztlich die zu ermittelnde \emph{Orientierung} des Kopfes exakter bestimmt werden kann. In den folgenden Abschnitten werden unterschiedliche Varianten der Filter-Fusionierung dargestellt und entsprechend der Aufgabenstellung des Projekts bewertet. 


\subsubsection{Madgwick-Filter}

Der Madgwick-Filter gilt als neuartiger Ansatz zur Bestimmung der Orientierung basierend auf 3D-Gyroskop, Beschleunigungssensor und 3D-Magnetometer \cite{}.

Auch wenn der Madgwick-Filter bereits als fertige ROS-Node vorhanden ist und  bessere Ergebnisse als der Kalman-Filter liefern soll, haben wir uns gegen den Madgwick-Filter entschieden, da das Optimierungsverfahren auf eine Darstellung über Quaternione basiert und dadurch der erzielte Mehrwert und erreichten Verbesserungen so nur schwer nachvollziehbar bzw. debuggbar gewesen sind. Außerdem war der Madgwick-Filter nur schwer an unsere bisherige Implementierung anpassbar.

\subsubsection{Kalman-Filter}
Im ersten Schritt \emph{(Prädiktionsschritt)} des Kalman-Filters wird die zeitlich vorangegangene Schätzung als Grundlage genommen, um die Vorhersage für den aktuellen Zeitpunkt zu ermitteln. Der zweite Schritt \emph{(Korrektur- oder Innovationsschritt)} besteht daraus, die Vorhersagen mit neuen Informationen des aktuelle Messwerts zu korrigieren um somit die gesuchten Schätzwerte zu erreichen.

Um bei einer fehlerbehafteten Beobachtung den Systemzustand zu korrigieren müssen fehlerhafte Beobachtungen über mathematische Gleichungen beschreibbar und Fehlerverteilungen für Sensoren bekannt sein. Diese Informationen sind uns während der Bearbeitung des Projekts nicht bekannt gewesen. Infolgedessen haben wir beschlossen den hierfür benötigten Mehraufwand an Zeit und Ressourcen lieber in den bereits fortgeschrittenen Komplementär-Filter zu investieren.

\subsubsection{Komplementär-Filter}
Nach Brooks und Iyengar \cite{} \todo{Cite} hat eine komplementäre Fusion das Ziel, die Genauigkeit von Daten zu verbessern. Dabei wirken die Sensoren unabhängig voneinander und liefern unterschiedliche Erkenntnisse und Sichtbereiche die Orientierung zu unterschiedlichen Zeiten verbessert.

Die Gyroskope sind wie bereits erwähnt bei einer langen Laufzeiten fehleranfällig und erzeugen einen drift. Daher wird im ersten Schritt eine lineare Interpolation (engl. slerp) eingesetzt, die bei jedem Zeitschritt die Orientierung der Gyroskope zu $95\%$ und Orientierung des Beschleunigungssensor zu $5\%$ berücksichtigt. Hierbei wird das \emph{roll} und \emph{pitch} gestützt. Der zweite Schritt zielt auf die Verbesserung von \emph{yaw} ab. Eine weitere Interpolation verrechnet dieses Ergebnis mit den aufbereiteten Magnetometerwerten zu $5\%$ pro Zeitschritt. Formal lässt sich die Gewichtung folgendermaßen beschreiben:
\begin{equation}
    gyro.slerp(0.05, acc).slerp(0.05, mag)
\end{equation}


Eine Übersicht der Fusionierung mit dem Komplementär-Filter ist Abbildung \ref{fig:fusion_complementary} zu entnehmen.

\begin{figure}[ht]
   \centering
   \includegraphics[width=0.5\textwidth]{fusion_complementary.png}
   \caption[]{Fusion -- Komplementärfilter}
   \label{fig:fusion_complementary}
\end{figure}

\todo[inline]{Off the topic...}

Vorteile:
Sehr einfacher Filter
gut genug für unsere Zwecke
leicht anpassbar
Blockschaltbild der Fusion
Gewichtung: Funktionsweise formal: “gyro.slerp(0.05, acc).slerp(0.01, mag)”


